\section{Module \& Detector Development}
\subsection{Implementing a New Module}
\label{sec:building_new_module}
Before starting to implement your own module it is highly recommended to go through the framework module documentation in section \ref{sec:module_manager}. The information in \ref{sec:module_files} and \ref{sec:module_structure} is essential before you can start creating your own module. Thereafter one should roughly follow the steps below to start creating your own module \texttt{ModuleName} (constantly replacing  \texttt{ModuleName} with the real name of your module):
\begin{enumerate}
\item Copy the whole directory \textit{src/modules/DummyModule/} to \textit{src/modules/\texttt{ModuleName}/}.
\item Rename DummyModule.hpp to \texttt{ModuleName}Module.hpp and DummyModule.cpp to \texttt{ModuleName}Module.cpp.
\item Determine if you want to make a unique or detector-specific module and modify the CMakeLists.txt to conform to this choice. Also fix the name of the source file to the name determined above.
\item Modify the header and source files to follow the constructor conventions for the specific type of module.
\item Start filling up the initial documentation parts of the README.md and the \LaTeX-file \texttt{ModuleName}.tex.
\item Fill the constructor, \texttt{init}, texttt{run} and/or \texttt{finalize} method depending on what your module should do.
\item Add your module to the build process by adding option\_build\_module(\texttt{ModuleName}) at the appropriate location in \textit{src/modules/CMakeLists.txt}
\end{enumerate}
\todo Part of the process above can be automized in a script \todo

After this, it is up to the developer to make a module as advanced as necessary (do however consider that at some point it may be beneficial to split up modules to keep the modular design intact). Sources which may be primarily useful during the development of the module include:
\begin{itemize}
\item The framework documentation in section \ref{sec:framework} for a gentle introduction to the different parts in the framework.
\item The module documentation in section \ref{sec:modules} for a description of functionality other modules already provide and to look for similar modules which can help during development.
\item The Doxygen (core) reference documentation included in the framework (\todo how to build this / where to find this ? \todo).
\item The latest version of the source code of all the modules (and the core itself). Freely available to copy and modify under the MIT license at \url{https://gitlab.cern.ch/simonspa/allpix-squared/tree/master}.
\end{itemize}

In case you think a module may be useful for other people, feel free to send a merge-request to the main repository at \url{https://gitlab.cern.ch/simonspa/allpix-squared/merge_requests}.

\subsection{Adding a New Detector Model}
\label{sec:adding_detector_model}
\todo The format for the detector models should be discussed and approved before this section can be finished \todo
