\section{Objects}
\label{sec:objects}

\subsection{Object History}
\label{sec:objhistory}

\todo{Add description of object history using TRef, vector pointers to ancesters, necessity to call the "depends" message binding as soon as it is there for having the respective objects at hand and not dropped}

\subsection{Object Types}
\label{sec:objtypes}

\apsq provides a set of objects that should be used to transfer data between modules.
These objects can be send with the messaging system as explained in Section \ref{sec:objects_messages}.
A \texttt{typedef} is added to every object in order to provide an alternative name for the message which is directly indicating the carried object.

The list of currently supported objects comprises:
\nlparagraph{MCParticle}
The Monte-Carlo truth information about the particle passage through the sensor.
Both the entry and the exit point are stored in the object, to approximate the track.
The exact handling of non-linear tracks due to multiple scattering, is up to module.
The MCParticle also stores an identifier of the particle type.
The recommended naming scheme are the PDG particle codes~\cite{pdg}.

\nlparagraph{DepositedCharge}
Set of charge carriers deposited by an ionizing particle crossing the active material of the sensor.
The object stores the \underline{local} position in the sensor together with the total number of deposited charges in elementary charge units.
Also the time (in \textit{ns} as the internal framework unit) of the deposition after the start of the event and the type of carrier (electron or hole) is stored.

\nlparagraph{PropagatedCharge}
Set of charge carriers propagated through the silicon sensor due to drift and/or diffusion processes.
The object should store the final \underline{local} position of the propagation.
This is either on the pixel implant if the set of charge carriers are ready to be collected, or on any other position in the sensor if the set of charge carriers got trapped or was lost in another process.
Timing information about the total time to arrive at the final location, from the start of the event, can also be stored.

\nlparagraph{PixelCharge}
Set of charge carriers collected at a single pixel.
The pixel indices are stored in both the $x$ and $y$ direction, starting from zero for the first pixel.
Only the total number of charges at a pixel is currently stored, the timing information of the individual charges can be retrieved from the related PropagatedCharge objects.

\nlparagraph{PixelHit}
Digitized hit of a pixel after processing of the information in the detector's front-end electronics.
The object allows to store both the time and a signal value.
The time can be stored in an arbitrary unit used to timestamp the hits.
The signal can hold different kinds of information depending on the type of the digitizer used.
Examples of the signal information is the 'true' information of a binary readout chip, the number of ADC counts or the ToT (time-over-threshold).
