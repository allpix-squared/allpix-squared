\section{Getting Started}
This section serves as a general introduction to \apsq for users. It is constructed to focus on the basics to get everything up\&running as quickly as possible. More details about the framework and other implementation details are available in section \ref{sec:framework}.

In this section it is assumed that the installation of \apsq as described in section \ref{sec:installation} has been succesfull. Also this getting started guide is written with a default installation in mind, this means that some parts of it may not work if you decided to do a custom installation. When the \textit{allpix} binary is called, the executable installed in \text{bin/allpix} in your installation path is meant. Remember that before running any \apsq simulation, ROOT and likely Geant4 should be initialized. Refer to section \ref{sec:initialize_dependencies} on instructions how to include those libraries.

\subsection{Configuration Files}
\label{sec:configuration_files}
The framework can be configured with configuration files. The configuration format is described in detail in section \ref{sec:config_file_format}, but the format should be straightforward to use. The configuration consists of several section headers specified within $[$ and $]$ brackets (and the default empty section at the start). Every section contain a set of key/value pairs separated by the \texttt{=} character. The \texttt{\#} is used to indicate comments.

The framework has three required configuration layers which are the following:
\begin{itemize}
\item The \textbf{main} configuration: The most important configuration file and the file that is passed directly to the binary. Contains both the global framework configuration and the list of modules to instantiate together with their configuration. More details and an example are found in section \ref{sec:main_config}.
\item The \textbf{detector} configuration passed to the geometry builders \todo{Should be passed to initial geometry reader instead?}. Describes the detector setup usually referred as the telescope. Contains the position, orientation and model type of all detectors. Introduced in section \ref{sec:detector_config}.
\item The detector \textbf{models} configuration. Contain the parameters describing a particular type of detector. These are shipped by the framework and one should typically not need to change these or add new models. If this is however needed, please refer to \ref{sec:adding_detector_model} for more details about adding new models.
\end{itemize}

Before going into depth on defining the required configuration files, first more detailed information about specifying the configuration values is provided.

\subsubsection{Supported types and units}
\label{sec:config_values}
The \apsq framework supports the use of a variety of types for all configuration values. The module specifies how the value type should be interpreted. An error will be raised if either the key is not specified in the config file, the conversion to the desired type is not possible or if the value is non-sense for that particular key for any other reason. Please refer to the module documentation in section \ref{sec:modules} for the list of module parameters and their types. Parsing the value roughly follows common-sense (more details can be found in section \ref{sec:accessing_parameters}). There do apply a few special rules:
\begin{itemize}
\item If the value is a \textbf{string} it may be enclosed by a single pair of double quotation marks (\texttt{"}), which are stripped before passing the value to the modules. If the string is not enclosed by the quotation marks all whitespace before and after the value is erased. If the value is an array of strings, the value is split at every whitespace or comma (\texttt{'}) that is not enclosed in quotation marks.
\item If the value is a \textbf{relative path} that path will be made absolute by adding the absolute path of the directory that contains the configuration file where the key defined.
\item If the value is an \textbf{arithmetic} type that arithmetic value may have a suffix indicating the unit. The list of base units is shown in Table \ref{tab:units}. 
\end{itemize}

\begin{table}[h]
\centering
\caption{List of units supported by \apsq}
\label{tab:units}
\begin{tabular}{|l|l|l|}
\hline
\textbf{Quantity}                 & \textbf{Default unit}                   & \textbf{Auxiliary units} \\ \hline
\multirow{6}{*}{\textit{Length}}  & \multirow{6}{*}{mm (millimeter)}        & nm (nanometer)           \\ \cline{3-3} 
                                  &                                         & um (micrometer)          \\ \cline{3-3} 
                                  &                                         & cm (centimeter)          \\ \cline{3-3} 
                                  &                                         & dm (decimeter)           \\ \cline{3-3} 
                                  &                                         & m (meter)                \\ \cline{3-3} 
                                  &                                         & km (kilometer)           \\ \hline
\multirow{4}{*}{\textit{Time}}    & \multirow{4}{*}{ns (nanosecond)}        & ps (picosecond)          \\ \cline{3-3} 
                                  &                                         & us (microsecond)         \\ \cline{3-3} 
                                  &                                         & ms (millisecond)         \\ \cline{3-3} 
                                  &                                         & s (second)               \\ \hline
\multirow{4}{*}{\textit{Energy}}  & \multirow{4}{*}{MeV (megaelectronvolt)} & eV (electronvolt)        \\ \cline{3-3} 
                                  &                                         & keV (kiloelectronvolt)   \\ \cline{3-3} 
                                  &                                         & GeV (gigaelectronvolt)   \\ \hline
\textit{Temperature}              & K (kelvin)                              &                          \\ \hline
\textit{Charge}                   & e (elementary charge)                   & C (coulomb)              \\ \hline
\multirow{2}{*}{\textit{Voltage}} & \multirow{2}{*}{MV (megavolt)}          & V (volt)                 \\ \cline{3-3} 
                                  &                                         & kV (kilovolt)            \\ \hline
\textit{Angle}                    & rad (radian)                            & deg (degree)             \\ \hline
\end{tabular}
\end{table}

Combinations of base units can be specified by using the multiplication sign \texttt{*} and the division sign \texttt{/} that are parsed in linear order (thus $\frac{V m}{s^2}$ should be specified as $V*m/s/s$). The framework assumes the default units (as given in Table \ref{tab:units}) if the unit is not explicitly specified. It is recommended to always specify the unit explicitly for all parameters that are not dimensionless (and also for angles which are special units). 

Examples of specifying key/values pairs of various types are given below
\begin{minted}[frame=single,framesep=3pt,breaklines=true,tabsize=2,linenos]{ini}
# all white at the front and back is removed
first_string = string_without_quotation
# all whitespace within the quotation marks is kept
second_string = "  string with quotation marks  "
# keys are splitted on whitespace and commas
string_array = "first element" "second element","third element"
# integer and floats can be given in standard formats
int_value = 42
float_value = 123.456e9
# units can be passed to arithmetic type
energy_value = 1.23MeV
time_value = 42ns
# units are combined in linear order
acceleration_value = 1.0m/s/s
# thus the quantity below is the same as deg*kV*K/m/s
random_quantity = deg*kV/m/s*K 
# relative paths are expanded to absolute 
# path below will be /home/user/test if the config file is /home/user
output_path = "test" 
\end{minted}

\subsubsection{Detector configuration}
\label{sec:detector_config}
The detector configuration consist of a set of section headers describing the detectors in the setup. The section header describes the name of the detector. All names should be unique, thus using the same name multiple times is not allowed. Every detector should contain all of the following parameters:
\begin{itemize}
\item A string referring to the \textbf{type} of the detector model. The model should exist in the search path described in section \ref{sec:detector_models}.
\item The 3D \textbf{position} in the world frame in the order x, y, z. See section \ref{sec:models_geometry} for details.
\item The \textbf{orientation} specified as Z-X-Z Euler angle relative to the world axis. See section \ref{sec:models_geometry} for details.
\end{itemize}

An example configuration file for two test detectors and one timepix is given below. 
\begin{minted}[frame=single,framesep=3pt,breaklines=true,tabsize=2,linenos]{ini}
# name the first detector test1
[test1]
# set the type to test
type = "test"
# place it at the origin of the world frame
position = 0 0 0
orientation = 0 0 0

# name the second detector test2
[test2]
# set the type again to test
type = "test"
# place it 0.5 mm down on the z-axis from the origin
position = 0 0 -0.5mm
# use the default orientation
orientation = 0 0 0

# name the third detector tpx
[tpx]
# set the type to timepix
type = "timepix"
# set the position in the world frame
position = 10um 10um -0.25mm
# rotate 45 degrees around the world z-axis
orientation = 45deg 0 0
\end{minted}
This configuration file is used in the rest of this chapter for explaining concepts. 

\subsubsection{Main configuration}
\label{sec:main_config}
The main configuration consists of a set of section header that specify the modules used. All modules are executed in the \underline{linear} order in which they are defined. Modules can be specified multiple times in the configuration files, but it depends on their type and configuration if that is allowed or not, as explained later on. There are a few section names that have a special meaning in the main configuration, which are the following:
\begin{itemize}
\item The \textbf{global} (framework) header sections: These are all the zero-length section headers (including the one at the start) and all with the header \texttt{AllPix} (case-sensitive). These are combined and accessed together as the global configuration, which contain all the parameters of the framework (see section \ref{sec:framework_parameters} for details). All key-value pairs defined in this section are also inherited in all individual configurations as long the key is not defined in the specific configuration.
\item The \textbf{ignore} header sections: All the section with name \texttt{Ignore} are fully ignored. All key-value pairs as well as the section itself are threated as if they were never defined. These sections are useful for quickly enabling and disabling certain sections.
\end{itemize}

The rest of all the section headers are used to instantiate all the modules. Modules should normally be loaded automatically (as long as they are installed correctly). If problems arise please check the loading rules in section \ref{sec:module_instantiation}. The type of the module determines how the module is instantiated:
\begin{itemize}
\item If the module is \textbf{unique} it runs only a single time. These kind of modules should only appear once in the whole configuration file.
\item If the module is \textbf{detector}-specific it is run on every specified detector it is specified to run on. By default an instantiation is created for all detectors defined in the detector configuration file (see section \ref{sec:detector_config}) unless one or both of the following parameters are specified.
\begin{itemize}
\item \textbf{name}: An array of detector names where the module should run on. Replaces all global and type specific modules of the same kind.
\item \textbf{type}: An array of detector type where the module should run on. Instantiated after considering all detector specified by the name parameter above. Replaces all global modules of the same kind. 
\end{itemize}
\end{itemize}

A valid example configuration using the detector configuration above could be:
\begin{minted}[frame=single,framesep=3pt,breaklines=true,tabsize=2,linenos]{ini}
# key is part of the empty section and therefore the global sections
string_value = "example1"

# the AllPix section is a special global section and merged with above
[AllPix]
another_random_string = "example2"

# first run a unique module
[MyUniqueModule]
# this module takes no parameters

# [MyUniqueModule] cannot be instantiated another time

# then run some detector modules on different detectors 
# first run a module on the two detector with name test
[MyDetectorModule]
type = "test"
int_value = 1

# replace the module above for test1 with a specialized version 
[MyDetectorModule]
name = "test1"
int_value = 2

# runs the module on the remaining unspecified detector tpx
[MyDetectorModule]
# int_value is not specified, so it uses the default value
\end{minted}
This configuration can however not be run in practice because MyUniqueModule and MyDetectorModule do not actually exist. Let us therefore continue to look at some actual parameters and start constructing a useful configuration file. In the next section the global framework parameters are introduced first.

\subsection{Framework parameters}
\label{sec:framework_parameters}
The framework has a variety of global parameters that allow to configure \apsq to your own needs. The current list of global parameters (roughly sorted from most important to least important for typical simulations):
\begin{itemize}
\item \textbf{number\_of\_events}: Determines the amount of events the framework should simulate. Equivalent to the amount of times the modules are run. Defaults to one (simulating a single event).
\item \textbf{log\_level}: Specifies the minimum log level to write to the screen. Possible values include QUIET, FATAL, ERROR, WARNING, INFO and DEBUG, all options are case-insensitive. Defaults to the INFO level. More details and information about changing the log level for a particular module can be found in section \ref{sec:logging_verbosity}.
\item \textbf{log\_format}: Determines the format to display. Possible options include SHORT, DEFAULT and DEBUG. More information again in section \ref{sec:logging_verbosity}.
\item \textbf{output\_directory}: Top directory to write all output files into. Extra directories are created for all the module instantiations. Defaults to the current working directory with the subdirectory \textit{output/} attached.
\item \textbf{random\_seed}: Seed to use for the random seed generator (see section \ref{sec:random_generator}). A random seed will be used from multiple entropy sources if the parameter is not specified. Can be used to redo an earlier simulation run.
\item \textbf{library\_directories}: Additional directories to search for libraries. See section \ref{sec:module_instantiation} for details.
\item \textbf{model\_path}: Additional detector models to read. Refer to section \ref{sec:detector_models} for more information \todo{not strictly a framework parameter now, but inherited in the readers}.
\end{itemize}

With this information in mind it is time to setup a real simulation. All the module parameters are shortly introduced when they are first used. For more details about these parameters the module documentation in section \ref{sec:modules} should be checked.

\subsection{Setting up the Simulation Chain}
Let us directly start with a full, but simple, simulation. A typical simulation in \apsq contains at least the following components.
\begin{itemize}
\item The \textbf{geometry builder}, responsible for creating the global internal geometry (Geant4).
\item The \textbf{deposition} module that deposits the charges in the detectors using the provided physics list and the global geometry created above.
\item A \textbf{propagation} module that propagates the charges in to the sensor implant and transfers it to a pixel.
\item A \textbf{digitizer} module which converts the charges in the pixel to a detector hit
\item An \textbf{output} module, saving the data of the simulation to the requested format.
\end{itemize}
\todo{An extra geometry reader should be added here to create the \apsq geometry}

In the example we are going to deposit charges in all of the three sensors from the detector configuration file in \ref{sec:detector_config}. Only the charges in the timepix module are going to be propagated and digitized. The final results of hits in the timepix will be written to a ROOT histogram. The documented version of this main configuration file (with all the required parameters) is as follows:
\begin{minted}[frame=single,framesep=3pt,breaklines=true,tabsize=2,linenos]{ini}
# initialize the global configuration
[AllPix]
# run a total of 10 events
number_of_events = 10
# use the short logging format
log_format = "SHORT"

# read and instantiate the detectors and construct the Geant4 geometry
[GeometryBuilderGeant4]
# location of the detector configuration
detectors_file = "detector_config.ini"
# size of the world 
# TODO: this will be optional
world_size = 1000um 1000um 2000um

# initialize physics list, setup the particle source and deposit the charges
[DepositionGeant4]
# use one of the standard Geant4 physics lists
physics_list = QGSP_BERT
# use a charged pion as particle
particle_type = "pi+"
# use a single particle in a single 'event'
particle_amount = 1 
# set the energy of the particle 
particle_energy = 120GeV
# the position of the point source
particle_position = 0 0 0.2mm
# the shooting direction of the source (negative z-axis)
particle_direction = 0 0 -1

# read an electric field
# TODO: this should be excluded in this basic example
[ElectricFieldReaderInit]
file_name = "example_electric_field.init"

# propagate the charges in the sensor 
[SimplePropagation]
# only propagate the timepix
type = "timepix"
# set the temperature of the sensor
temperature = 293K
# propagate multiple charges together in one step to speed up
charge_per_step = 250

# transfer the propagated charges to the pixels
# TODO: this module is going to be deleted
[SimpleTransfer]
# maximum distance from the bottom to become mapped
max_depth_distance = 10um 

# save histogram for the timepix to an output file
[detector_histogrammer_test]
# alternative way to specify to run this only for the timepix
name = "tpx"
\end{minted}

The configuration above can be saved to \textit{config.ini} \todo{which suffix to use}. The detector configuration file in \ref{sec:detector_config} should then be saved as \textit{detector\_config.ini} in the same directory. Hereafter the total simulation can be executed by passing the configuration to the allpix binary as follows:

\begin{verbatim}
$ allpix config.ini
\end{verbatim}

The simulation should then start. It should return similar output to the one below.
\begin{lstlisting}[breaklines]
(I) Loading library libAllpixModuleGeometryBuilderGeant4.so
(I) [C:GeometryBuilderGeant4] Constructing internal geometry
(I) [C:GeometryBuilderGeant4] Reading model files
(I) [C:GeometryBuilderGeant4] Parsing models
(I) Loading library libAllpixModuleDepositionGeant4.so
(I) Loading library libAllpixModuleElectricFieldReaderInit.so
(I) Loading library libAllpixModuleSimplePropagation.so
(I) Loading library libAllpixModuleSimpleTransfer.so
(I) Loading library libAllpixModuledetector_histogrammer_test.so
(I) [I:GeometryBuilderGeant4] Building Geant4 geometry
(I) [I:DepositionGeant4] Initializing physics processes
(I) [I:DepositionGeant4] Constructing particle source
(I) [I:ElectricFieldReaderInit:test1] Reading electric field file
(W) [I:ElectricFieldReaderInit:test1] thickness of sensor in file is 0.285 but in the model it is 0.04
(W) [I:ElectricFieldReaderInit:test1] pixel size is (0.15,0.1) but in the model it is (0.022,0.044)
(I) [I:ElectricFieldReaderInit:test2] Reading electric field file
(W) [I:ElectricFieldReaderInit:test2] thickness of sensor in file is 0.285 but in the model it is 0.04
(W) [I:ElectricFieldReaderInit:test2] pixel size is (0.15,0.1) but in the model it is (0.022,0.044)
(I) [I:ElectricFieldReaderInit:tpx] Reading electric field file
(W) [I:ElectricFieldReaderInit:tpx] thickness of sensor in file is 0.285 but in the model it is 0.3
(W) [I:ElectricFieldReaderInit:tpx] pixel size is (0.15,0.1) but in the model it is (0.055,0.055)
(I) [I:detector_histogrammer_test:tpx] Creating histograms
(I) Running event 1 of 10
(I) [R:DepositionGeant4] Enabling beam
(I) [R:SimplePropagation:tpx] Propagating charges in sensor
(I) [R:SimpleTransfer:tpx] Transferring charges to pixels
(I) [R:SimpleTransfer:tpx] Combining charges at same pixel
(I) Running event 2 of 10
(I) [R:DepositionGeant4] Enabling beam
(I) [R:SimplePropagation:tpx] Propagating charges in sensor
(I) [R:SimpleTransfer:tpx] Transferring charges to pixels
(I) [R:SimpleTransfer:tpx] Combining charges at same pixel
(I) Running event 3 of 10
(I) [R:DepositionGeant4] Enabling beam
(I) [R:SimplePropagation:tpx] Propagating charges in sensor
(I) [R:SimpleTransfer:tpx] Transferring charges to pixels
(I) [R:SimpleTransfer:tpx] Combining charges at same pixel
(I) Running event 4 of 10
(I) [R:DepositionGeant4] Enabling beam
(I) [R:SimplePropagation:tpx] Propagating charges in sensor
(I) [R:SimpleTransfer:tpx] Transferring charges to pixels
(I) [R:SimpleTransfer:tpx] Combining charges at same pixel
(I) Running event 5 of 10
(I) [R:DepositionGeant4] Enabling beam
(I) [R:SimplePropagation:tpx] Propagating charges in sensor
(I) [R:SimpleTransfer:tpx] Transferring charges to pixels
(I) [R:SimpleTransfer:tpx] Combining charges at same pixel
(I) Running event 6 of 10
(I) [R:DepositionGeant4] Enabling beam
(I) [R:SimplePropagation:tpx] Propagating charges in sensor
(I) [R:SimpleTransfer:tpx] Transferring charges to pixels
(I) [R:SimpleTransfer:tpx] Combining charges at same pixel
(I) Running event 7 of 10
(I) [R:DepositionGeant4] Enabling beam
(I) [R:SimplePropagation:tpx] Propagating charges in sensor
(I) [R:SimpleTransfer:tpx] Transferring charges to pixels
(I) [R:SimpleTransfer:tpx] Combining charges at same pixel
(I) Running event 8 of 10
(I) [R:DepositionGeant4] Enabling beam
(I) [R:SimplePropagation:tpx] Propagating charges in sensor
(I) [R:SimpleTransfer:tpx] Transferring charges to pixels
(I) [R:SimpleTransfer:tpx] Combining charges at same pixel
(I) Running event 9 of 10
(I) [R:DepositionGeant4] Enabling beam
(I) [R:SimplePropagation:tpx] Propagating charges in sensor
(I) [R:SimpleTransfer:tpx] Transferring charges to pixels
(I) [R:SimpleTransfer:tpx] Combining charges at same pixel
(I) Running event 10 of 10
(I) [R:DepositionGeant4] Enabling beam
(I) [R:SimplePropagation:tpx] Propagating charges in sensor
(I) [R:SimpleTransfer:tpx] Transferring charges to pixels
(I) [R:SimpleTransfer:tpx] Combining charges at same pixel
(I) [F:detector_histogrammer_test:tpx] Writing histograms to file
\end{lstlisting}
The final histogram output should be availabe in \\ \textit{output/detector\_histogrammer\_test\_tpx/histogram.root} \todo{Both the directory and the name of this module should be reworked}.

If problems occur, please make sure you have an up-to-date and properly installed version of \apsq (see the installation instructions in section \ref{sec:installation}). If modules and models fail to load, more information about loading problems can be found in the detailed framework section \ref{sec:framework}.

\subsubsection{Adding new modules}
Before going to more advanced configurations, a few modules are discussed which a user might want to add.

\paragraph{VisualizationGeant4}
Displaying the geometry and the particle tracks helps a lot in both checking and interpreting the results. Visualization options are currently limited, but a simple visualization module is provided through Geant4. It supports all the visualization options provided by Geant4 as provided in their manual at \url{https://geant4.web.cern.ch/geant4/UserDocumentation/UsersGuides/ForApplicationDeveloper/html/ch08.html}. The VRML viewer is however the recommended option (until better solutions are available). 

To start using the VRML viewer, follow the steps below:
\begin{itemize}
\item First the \texttt{simple\_view = 1} parameter should be added to the \textbf{GeometryBuilderGeant4} section. This simplifies the visualization output and makes it several order of magnitude faster. 
\item Then the viewer should be installed on your operating system. Options include FreeWRL (available on \url{http://freewrl.sourceforge.net/}) and OpenVRML (available on \url{http://openvrml.org/}).
\item Subsequently two environmental parameters should be exported to inform Geant4 of the configuration. These include \texttt{G4VRMLFILE\_VIEWER} which should point to the location of the viewer and \texttt{G4VRMLFILE\_MAX\_FILE\_NUM} which should typically be set to 1 to prevent too many files from being created.
\item Finally the example section below should be added at the end of the configuration file before running the simulation again:
\end{itemize}

\begin{minted}[frame=single,framesep=3pt,breaklines=true,tabsize=2,linenos]{ini}
[VisualizationGeant4]
# use the VRML driver
driver = "VRML2FILE" 
# point those macro files to the etc/ directory in the allpix source folder
# TODO: these macros should be optional
macro_init = "geant4_init.in"
macro_run = "geant4_run.in"
\end{minted}

\paragraph{ElectricFieldReaderInit}
An electric field in the detectors can be specified using the .init format by using the ElectricFieldReaderInit module. The init format is a temporary format, taken from the PixelAV software\cite{pixelavgit} \todo{adding a more detailed description is useless as this will be removed anyway?}. These fields can be attached to specific detectors using the standard syntax for detector binding. An example would be:
\begin{minted}[frame=single,framesep=3pt,breaklines=true,tabsize=2,linenos]{ini}
[ElectricFieldReaderInit]
# bind the electric field to the timepix sensor
name = "tpx"
# name of the file containing the electric field
file_name = "example_electric_field.init"
\end{minted}
An example electric field (which the name used above) can be found in the \textit{etc} directory of the \apsq source folder.

\subsubsection{Advanced configuration}
\paragraph{Redirect Module Inputs and Outputs}
\wip

\paragraph{Using TCAd Electric Field Simulations}
\wip

\paragraph{Choosing the Propagation Modules}
\wip

\subsection{Logging and Verbosity Levels}
\label{sec:logging_verbosity}
One of the main design goals of \apsq is it being easy-to-use. One of the most important elements determining the user-friendlyness is the amount of feedback provided during the simulation. \apsq is designed to catch mistakes and implementation errors as early as possible and tries to give the user a clear indication about the problem. More details about the logger and the internal structure for reporting errors can be found in section \ref{sec:logger} and \ref{sec:error_reporting_exceptions}. This section will instead provide a quick introduction to the logging and information how to set the verbosity levels to get the details you require.

\todo{The logger should be restructured before this section can be finished}

\subsection{Storing Output Data}
\wip
