%----------------------------------------------------------------------------------------
%   PACKAGES AND DOCUMENT CONFIGURATIONS
%----------------------------------------------------------------------------------------

\documentclass{scrartcl}

\usepackage{siunitx} % Provides the \SI{}{} command for typesetting SI units
\usepackage{booktabs}

\usepackage{hyperref}
\usepackage{xspace}
\usepackage{color}
\usepackage{minted}

\usepackage{graphicx} % Required for the inclusion of images
\usepackage{listings}
\setlength\parindent{0pt} % Removes all indentation from paragraphs

\renewcommand{\labelenumi}{\alph{enumi}.} % Make numbering in the enumerate environment by letter rather than number (e.g. section 6)
\newcommand{\apsq}{\texorpdfstring{\ensuremath{\mathrm{allpix}^2}}{allpix\textasciicircum 2}\xspace}

%----------------------------------------------------------------------------------------
%   DOCUMENT INFORMATION
%----------------------------------------------------------------------------------------

\def\etal {\itshape{et~al.}}


\title{\apsq User Manual} % Title

\author{Koen Wolters (\href{mailto:koen.wolters@cern.ch}{koen.wolters@cern.ch})\\
  Simon Spannagel (\href{mailto:simon.spannagel@cern.ch}{simon.spannagel@cern.ch})
} % Author names

\date{\today} % Date for the report

\begin{document}

\maketitle % Insert the title, author and date


\begin{abstract}
\end{abstract}

\clearpage
% Table Of Contents
\tableofcontents

\clearpage

\section{Introduction}

\section{Installation}
\subsection{Prerequisites}
\subsection{Downloading the source code}
\subsection{Configuration via CMake}
\subsection{Compilation}

\section{The \apsq Framework}
\subsection{Architecture of the Framework}
\subsection{Modules and the Module Manager}
\subsection{Configuration and Parameters}
\subsection{Detector Models and Geometry}
\subsection{Passing Objects using Messages}
\paragraph{Object Types}
\subsection{Error Reporting and Exceptions}
\subsection{Logging and other Utilities}

\section{Configuration and Usage}
\subsection{Configuration Files}
\subsection{Framework Parameters}
\subsection{Setting up the Simulation Chain}
\paragraph{Add Additional Modules}
\paragraph{Redirect Module Inputs and Outputs}
\paragraph{Using TCAd Electric Field Simulations}
\paragraph{Choosing the Propagation Modules}
normal, irrad with trapping simulation etc
\subsection{Logging and Verbosity Levels}
\subsection{Storing Output Data}

\section{Module \& Detector Development}
\subsection{Implementing a New Module}
\subsection{Adding a New Detector Model}

\section{Frequently Asked Questions}
How do I run a module only for one detector?
How do I run a module only for a specific detector type?
...

\section{Additional Tools \& Resources}
\subsection{TCAD Electric Field Converter}
\subsection{Simple Usage Examples}

\section{Summary}

\section{Acknowledgments}
We would like to acknowledge the contribution of

\appendix
\clearpage
\input{usermanual/design.tex}
% probably some chapter with more advanced notes

\clearpage
\phantomsection
\addcontentsline{toc}{section}{References}
\begin{thebibliography}{99}
  \bibitem{allpixtwiki}
  M.~Benoit, \etal, ``Allpix Twiki Page'',
  available at \url{https://twiki.cern.ch/twiki/bin/view/Main/AllPix}.

\end{thebibliography}


\end{document}
